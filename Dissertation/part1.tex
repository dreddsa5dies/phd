\chapter{Теоретические основы автоматизации процесса работы наземных транспортно-технологических средств}\label{ch:ch1}

\section{Обзор текущего состояния автоматизации в сфере наземных транспортно-технологических средств}\label{sec:ch1/sec1}

\subsection*{Определение автоматизации в наземном транспорте}

Формирование автоматики как самостоятельной отрасли науки и техники сопровождалось установлением определенных общепринятых понятий. Определенность понятий и их точное понимание имеют важное значение, так как методы и средства автоматики нашли широкое применение в различных отраслях народного хозяйства.

Автоматика -- отрасль науки и техники об управлении и контроле протекания различных процессов, действующих без непосредственного участия человека. Более конкретное (узкое) определение автоматики -- это совокупность методов и технических средств, исключающих участие человека при выполнении операций конкретного процесса.

Автоматизация -- процесс, при котором функции управления и контроля осуществляются методами и средствами автоматики. В применении к любому производству автоматизация характеризуется освобождением человека от непосредственного выполнения функций управления производственными процессами и передачей этих функций автоматическим устройствам. Понятие автоматизации имеет широкое содержание, включающее комплекс технических, экономических и социальных вопросов. Техническая направленность автоматизации позволяет организовать технологические процессы с такой скоростью, точностью, надежностью и экономичностью, которые человек обеспечить не может. Экономическая направленность позволяет получить сравнительно быструю окупаемость первоначальных затрат за счет снижения эксплуатационных расходов и повышения объема и качества выпускаемой продукции, а социальная направленность позволяет изменить характер и улучшить условия труда человека.

По степени автоматизации производства различают частичную, комплексную и полную автоматизацию.

Частичная автоматизация -- это автоматическое выполнение отдельных производственных операций, осуществляемое в тех случаях, когда определенные технологические процессы вследствие своей сложности или быстродействия невыполнимы для человека. Функции человека при частичной автоматизации определяются технологическим процессом и сводятся к участию в производственных операциях, контроле и управлении. Частично автоматизируется, как правило, действующее производственное оборудование, причем наиболее эффективно автоматизировать технологический процесс, который сравнительно легко можно функционально выделить из общего производства.

Комплексная автоматизация -- автоматическое выполнение всех основных производственных операций участка, цеха, завода, электростанции и т. д. как единого взаимосвязанного комплекса. Функции человека при комплексной автоматизации ограничиваются контролем и общим управлением. При комплексной автоматизации отдельные автоматические регуляторы и программные устройства должны быть связаны между собой, и образовывать единую систему управления.

Полная автоматизация -- высшая ступень, при которой автоматизируются все основные и вспомогательные участки производства, включая систему управления и контроля. Управление и контроль автоматизируются с помощью вычислительных машин или специализированных автоматических устройств. Функции человека при полной автоматизации сводятся к наблюдению за работой оборудования и устранению возникающих неисправностей.

При определении степени автоматизации следует учитывать прежде всего экономическую эффективность и техническую целесообразность в условиях конкретного производства.

В зависимости от выполняемых функций автоматизация классифицируется на следующие основные виды: управление, контроль, сигнализация, блокировка, защиты и регулирование.

Управление -- это совокупность действий, направленных на поддержание функционирования объекта в соответствии с заданной программой, выполняемых на основе определенной информации о значениях параметров управляемого процесса (приведенное определение термина «управление» имеет в основном технический смысл применительно к изучаемому предмету).

Любой процесс управления в каждый момент времени характеризуется одним или несколькими показателями, которые отражают физическое состояние управляемого объекта (температура, скорость, давление, электрическое напряжение, ток, электромагнитное поле и т. д.). Эти показатели в процессе управления должны изменяться по какому-либо закону или оставаться неизменными при изменении внешних условий и режимов работы управляемого устройства. Такие показатели называются параметрами управляемого процесса.

С точки зрения автоматизации производства управление разделяется на автоматическое и полуавтоматическое.

При автоматическом управлении подача команд на управляемый объект осуществляется от специальных устройств либо по заданной программе, либо на основании информации контролируемых параметров. При полуавтоматическом управлении контроль работы управляемого объекта и подачи команд осуществляется частично оператором. Полуавтоматическое управление может быть местным или дистанционным. При местном управлении аппараты управления и контроля размещаются рядом с объектом, при дистанционном -- на любом расстоянии от объекта.

Автоматический контроль -- автоматическое получение и обработка информации о значениях контролируемых параметров объекта с целью выявления необходимости управляющего воздействия. Автоматический контроль можно рассматривать как составную часть автоматического управления, так как для протекания процесса по заданной программе необходимо иметь информацию о значениях контролируемых параметров, с тем чтобы оказывать при необходимости управляющее воздействие. Контроль может быть непрерывным и дискретным. Непрерывный контроль -- это контроль, при котором контролируемые параметры постоянно сопоставляются с заданными значениями. Дискретный контроль -- это контроль, при котором сопоставление параметров осуществляется периодически. Контроль также классифицируется на местный и дистанционный. Местный контроль -- это контроль, при котором наблюдение за состоянием параметров осуществляется непосредственно у объекта, при дистанционном контроле наблюдение за состоянием параметров осуществляется на расстоянии от объекта.

Сигнализация -- это преобразование информации о функционировании контролируемого объекта (о значении характерных параметров) в условный сигнал, понятный дежурному или обслуживающему персоналу. Сигнализация обычно разделяется на технологическую и аварийную. Технологическая сигнализация извещает персонал о ходе процесса при возможных допустимых отклонениях контролируемых параметров. Извещение может быть в виде световых сигналов (загорание или мигание ламп, табло и т. д.), а также сочетанием световых и звуковых сигналов. Аварийная сигнализация извещает об отклонениях контролируемых параметров технологического процесса за допустимые пределы и необходимость вмешательства персонала. Аварийное извещение должно отличаться от .технологического по своему логическому восприятию. Обычно оно выполняется в виде световых и звуковых сигналов.

Пример технологической и аварийной сигнализации -- это функционирование релейной защиты электрической станции. При заданных значениях напряжения и тока постоянно горящее световое табло свидетельствует о нормальном режиме работы высоковольтного оборудования. При отклонении напряжения и тока электрической сети за допустимые значения срабатывает релейная защита и световое табло начинает мигать в сопровождении звуковых прерывистых сигналов.

Блокировка -- это фиксация механизмов, устройств в определенном состоянии в процессе их работы. Блокировка позволяет сохранить механизм, устройство в фиксированном положении после получения внешнего воздействия. Блокировка повышает безопасность обслуживания и надежность работы оборудования, обеспечивает требуемую последовательность включения механизмов, устройств, а также ограничивает перемещение механизмов в пределах рабочей зоны. Примером блокировки может служить устройство высоковольтного выключателя. Механизм блокировки устроен таким образом, что включение выключателя возможно только при закрытой лицевой панели.

Автоматическая защита -- это совокупность методов и средств, прекращающих процесс при возникновении отклонений за допустимые значения контролируемых параметров. Так, например, при перегрузках или коротких замыканиях в электрических сетях происходит срабатывание определенного вида защиты (тепловой, максимального тока и т. д.) и автоматическое отключение аварийных участков. В ряде случаев устройства защиты одновременно выполняют функции управления. Например, для повышения уровня бесперебойности электроснабжения защитные устройства с одновременным отключением аварийной цепи автоматически включают резервные цепи.

Автоматическое регулирование -- это автоматическое обеспечение заданных значений параметров, определяющих требуемое протекание управляемого процесса в соответствии с установленной программой. Автоматическое регулирование можно рассматривать как составную часть автоматического управления.

Параметры управляемого процесса, подлежащие заданным изменениям или стабилизации, называют регулируемыми параметрами.

Устройство, аппарат или изделие, у которых регулируются один или несколько параметров, называют объектом автоматического регулирования.

Устройство, обеспечивающее автоматическое поддержание заданного значения регулируемого параметра в управляемом объекте или его изменения по определенному закону, называют регулятором.

Совокупность объекта регулирования и автоматического регулятора называют системой автоматического регулирования (САР).

В системе автоматического регулирования различают прямую и обратную связь.

Прямая связь -- это воздействие каждого предыдущего элемента регулятора на последующий.

Обратная связь -- воздействие одного из последующих элементов регулятора на предыдущий. Обратная связь бывает положительной, когда направление ее воздействия совпадает с направлением воздействия предыдущего элемента на последующий, и отрицательной в противоположном случае.

Автоматизация — применение технических средств, экономико-математических методов и систем управления, освобождающих человека частично или полностью от непосредственного участия в процессах получения, преобразования, передачи и использования энергии, материалов или информации.

\subsection*{Роль и значение автоматизации для эффективности работы наземных средств}

Роль

\subsection*{Примеры основных принципов автоматизации}

Примеры

\section{Основные принципы автоматизации процесса работы наземных транспортно-технологических средств}\label{sec:ch1/sec2}

\subsection*{Системы управления и контроля}

По степени автоматизации различают машины с механизированным управлением, с автоматизированным управлением и контролем на базе микропроцессорной техники, с автоматизированным управлением на расстоянии, с автоматическим управлением на базе микропроцессоров и миниЭВМ, строительные манипуляторы и роботы, а также роботизированные машины и комплексы \cite[с.~39]{Evtukov}.

Системы управления предназначены для включения и выключения различных механизмов машин \cite[с.~109]{Evtukov}.
По назначению системы управления можно разделить на следующие: управлением двигателем; управление муфтами и тормозами; рулевое управление; управление рабочим органом (например, опускание и подъем отвала бульдозера или ковша скрепера, поворот отвала автогрейдера).

По конструкции системы управления строительных машин разделяют на механические, гидравлические, пневматические, электрические и смешанные (комбинированные), аналогично силовым приводам, но в отличие от которых в большинстве случаев в системах управления передаются значительно меньше силы.

Различают машины с механизированным и автоматизированным управлением. Автоматизированное управление и контроль рабочего процесса могут осуществляться на базе микропроцессоров и миниЭВМ, а также строительные манипуляторы и роботы, роботизированные машины и комплексы.

\subsection*{Использование датчиков и сенсоров}

Использование

\subsection*{Программное обеспечение для автоматизации}

По

\subsection*{Примеры инновационных технологий в наземном транспорте}

Сошлемся на библиографию и сделаем краткое сокращение \nomenclature{НТТС}{наземные транспортнотехнологические средства (машины и~комплексы)}

Одна ссылка: \cite[с.~39]{Evtukov}.
Две ссылки: \cite{doi:10.36652/1684-1298-2023-36-39,Baumana}.

\section{Технологии автоматизации в наземном транспорте}\label{sec:ch1/sec3}

\subsection*{Преимущества автоматизации для безопасности и эффективности работы}

Преимущества

\subsection*{Вызовы и проблемы, связанные с внедрением автоматизации}

Вызовы

\subsection*{Перспективы развития автоматизации в наземном транспорте}

Перспективы

\section{Выводы по главе}\label{sec:ch1/sec4}

Выводы

\FloatBarrier
