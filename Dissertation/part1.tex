\chapter{Теоретические основы автоматизации процесса работы наземных транспортно-технологических средств}\label{ch:ch1}

\section{Обзор текущего состояния автоматизации в сфере наземных транспортно-технологических средств}\label{sec:ch1/sec1}

- Определение автоматизации в наземном транспорте
- Роль и значение автоматизации для эффективности работы наземных средств
- Примеры основных принципов автоматизации

По степени автоматизации различают машины с механизированным управлением, с автоматизированным управлением и контролем на базе микропроцессорной техники, с автоматизированным управлением на расстоянии, с автоматическим управлением на базе микропроцессоров и мини-ЭВМ, строительные манипуляторы и роботы, а также роботизированные машины и комплексы. \cite[с.~39]{Evtukov}

\cite[с.~109]{Evtukov}
Системы управления предназначены для включения и выключения различных механизмов машин.
По назначению системы управления можно разделить на следующие: управлением двигателем; управление муфтами и тормозами; рулевое управление; управление рабочим органом (например, опускание и подъем отвала бульдозера или ковша скрепера, поворот отвала автогрейдера).
По конструкции системы управления строительных машин разделяют на механические, гидравлические, пневматические, электрические и смешанные (комбинированные), аналогично силовым приводам, но в отличе откоторых в большинтсве случаев в системах управления передаются значительно меньше силы.
Различают машины с механизированным и автоматизированным управлением. Автоматизированное управление и контроль рабочего процесса могут осуществляться на базе микропроцессоров и мини-ЭВМ, а также строительные манипуляторы и роботы, роботизированне машины и комплексы.

\section{Основные принципы автоматизации процесса работы наземных транспортно-технологических средств}\label{sec:ch1/sec2}

- Системы управления и контроля
- Использование датчиков и сенсоров
- Программное обеспечение для автоматизации
- Примеры инновационных технологий в наземном транспорте

Сошлёмся на библиографию и сделаем краткое сокращение \nomenclature{НТТС}{наземные транспортно-технологические средства (машины и~комплексы)}

Одна ссылка: \cite[с.~39]{Evtukov}.
Две ссылки: \cite{doi:10.36652/1684-1298-2023-36-39,Baumana}.

\section{Технологии автоматизации в наземном транспорте}\label{sec:ch1/sec3}

- Преимущества автоматизации для безопасности и эффективности работы
- Вызовы и проблемы, связанные с внедрением автоматизации
- Перспективы развития автоматизации в наземном транспорте

\section{Выводы по главе}\label{sec:ch1/sec4}

\FloatBarrier
